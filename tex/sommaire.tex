\section{Sommaire}
\label{s:sommaire}

Ce programme permet de transformer plusieurs objets différents par leur position, leur rotation, leur parent et autres.
On s’est grandement inspiré du fonctionnement de la scène du programme « Unity » pour créer cette application.\\

Évidemment, nous pouvons voir tous les objets qui sont dans la scène en bougeant la caméra.
Cette caméra peut bouger par translation et par rotation à l’aide de certaines touches du clavier.\\

Premièrement, nous avons le graphe de scène, dans lequel on retrouve tous les éléments existant dans celle-ci.
Nous avons la possibilité de créer des objets 3D (sphère et cube) ainsi qu’une variété d’objets 2D (tel qu’un cercle, un rectangle, une ligne, etc.).
Nous pouvons également créer 2 modèles 3D, soit le fameux Faucon Millénium et un vaisseau « XWing » de la série de films « Star Wars ».\\

Deuxièmement, nous avons l’inspecteur qui nous permet de modifier l’objet que nous avons sélectionné dans le graphe de scène.
Il nous permet de jouer avec la position, la rotation, l’échelle de grandeur, leur couleur (soit en RGBA ou en HSBA) et de lui définir son parent (ses valeurs seront changées selon celles de ce dernier). \\

Finalement, nous avons la fenêtre de texture, exclusive aux objets 2D.
Celle-ci nous permet de choisir une texture, ou aucune, parmi une variété de 5 choix (par exemple, un zoom).\\

Ce programme regroupe plusieurs fonctionnalités qui seront élaborées plus bas.
Le résultat est une application à allure professionnelle avec une interface utilisateur simple d’utilisation avec des contrôles de claviers et de souris intuitifs.