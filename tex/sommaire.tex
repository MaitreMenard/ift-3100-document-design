\section{Sommaire}
\label{s:sommaire}

Notre application permet de construire, éditer et rendre des scènes visuelles à l'aide d'une interface graphique et des raccourcis clavier.
Elle a été développée en C++ à l'aide de la librairie \emph{openFrameworks}.
Le programme repose sur une architecture orientée object fortement modulaire et découplée.
Nous nous sommes grandement inspirés du fonctionnement de l'éditeur de scène du programme « Unity » pour réaliser cette application.\\

Il est possible d'ajouter à la scène une vaste gamme de \emph{GameObjects}, puis de modifier les propriétés de ceux-ci.
Parmi les éléments qu'on peut créer, on retrouve des objets 3D, soient des sphères ou des cubes, et des objets 2D, tels que des cercles, des rectanles, des courbes paramétriques et autres.
On compte aussi des objets spéciaux, comme des modèles 3D de deux célèbres vaisseaux de la série « Star Wars », deux portails inspirés du jeu « Portals » ainsi qu'un mur de briques en relief. Tous les \emph{GameObjects} ajoutés à la scène font parti d'une structure de type graphe de scène. Cela permet à l'utilisateur de modifier la hiérarchie des objets afin de créer des relations parent-enfant.\\

L'aspect du rendu global de la scène peut être altéré par des effets pleine fenêtre.
On peut donc rendre la scène en noir et blanc, en tons de gris, en sépia, avec un flou gaussien et avec un effet cartoon inspiré des jeux vidéo « Borderlands », en plus du mode de rendu normal.
L'éclairage de la scène est également paramétrable et interagit avec les matériaux des objets.
Enfin, il est possible de modifier le point de vue dans la scène et changeant le mode de projection de la caméra ou en déplaçant celle-ci.
La caméra peut bouger par translation et par rotation à l’aide de certaines touches du clavier.\\

L'application possède une interface graphique très complète, tout en restant intuitive.
Différents panneaux offrent des contrôles permettant d'éditer la scène.
Le panneau du graphe de scène présente la liste des \emph{GameObjects} qui ont été créés et permet à l'utilisateur de sélectionner un objet.
L'inspecteur sert quant à lui à modifier le \emph{GameObjects} qui a été sélectionné.
Il nous permet de jouer avec différentes propriétés, dont leur nature dépend du type d'objet qui est sélectionné.
Parmi les attributs modifiables, on note la position, la rotation et l’échelle de grandeur, les propriétés du matériel, le numéro du \emph{GameObjects} parent et bien d'autres.
Enfin, deux autres panneaux s'affichent en fonction du type d'objet sélectionné, le premier servant à modifier le type de lumière et le second servant à appliquer une texture générée procéduralement, choisie parmi une variété de 5 choix, à un objet 2D.\\

Le résultat est une application d'allure professionnelle avec une interface graphique riche et simple d’utilisation ainsi que des contrôles de claviers et de souris intuitifs, qui peut rendre des scènes intéressantes.
Les détails sur le fonctionnement du logiciel seront élaborés dans les sections suivantes.
\clearpage
