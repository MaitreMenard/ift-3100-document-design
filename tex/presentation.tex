\section{Présentation}
\label{s:présentation}

\subsection{Maxime Ménard}
Je suis étudiant en quatrième année en génie informatique avec une concentration en systèmes intelligents.
J'ai contribué par le passé au projet Robocup Ulaval et je suis présentement responsable de l'équipe logicielle du Groupe Aérospatial de l'Université Laval (GAUL).
Dans ce projet, j'ai mis en place la structure du projet et l'architecture logicielle et j'ai travaillé sur la transformation et le rendu des objets.

\subsection{Alexandre Turcotte}
Je suis à ma première année universitaire, soit à ma deuxième session, en IFT avec une passerelle DEC-BAC.
Je suis également en concentration de jeux-vidéo, et donc plus tard mon but est d’être programmeur ou designer de jeux-vidéo.
J’ai fait plusieurs projets personnels, mon plus récent étant un clone de League of Legends qui me permettra de développer des personnages par moi-même et de les tester au lieu de simplement les faire sur papier.
Dans ce travail, j’ai principalement travaillé sur l’interface utilisateur et ses fonctionnalités ainsi que sur les relations parents-enfants.

\subsection{Julien Becirovski}
Je suis étudiant en dernière année de génie informatique avec une concentration en robotique mobile et leurs applications. À côté de mes études, j'ai participé à l'essort en 2013 du projet étudiant RoboCup ULaval, ainsi que le projet étudiant Véhicule Autonome ULaval en hiver 2017 dans lequel je suis responsable. J'ai choisi d'effectuer mon dernier cours de concentration en infographie dans le but de comprendre et d'appliquer les technologies liées au rendu 3D qui sont notamment utilisées dans les simulateurs.