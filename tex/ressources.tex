\section{Ressources}
\label{s:ressources}

La seule ressource originale produite par l'équipe est l'image \textit{grid2.png} située dans le dossier \textit{/bin/data}.
Il s'agit d'une grille 10x10 avec transparence qui a été réalisée dans \textit{Adobe Photoshop}.\\

Les modèles 3D du \textit{Faucon Millenium} ainsi que du \textit{X-Wing Fighter} utilisés dans l'application ont été trouvés sur le site \url{https://free3d.com/free-3d-models/obj}.
Le modèle du Faucon Millenium a été produit par A. Meerow et est la propriété de Disney/LucasFilm.
Le modèle du X-Wing Fighter a été produit par Glenn Campbell, Harry Chang, Jose Gonzales Pareja, Matt Walton ainsi que Matt Allen et est la propriété de Disney/LucasFilm.
Ces deux modèles 3D doivent être utilisés à des fins personnelles seulement.
Ils se trouvent dans le dossier \textit{/bin/data/models}.\\ 

Le code des textures procédurales s'inspire du site \textit{Texture Generation using Random Noise}, (Lode Vandevenne, 2004) \url{http://lodev.org/cgtutor/randomnoise.html}.
\label{src-procedural-texture}

Les shaders des modèles d'illumination (\ref{s:illumination}) proviennent de la solution du module 7 du cours \url{https://github.com/philvoyer/IFT3100H18/tree/master/module07/}, le shader pour le relief 
\ref{s:relief} avec les techniques de \textit{normal/bump mapping} est inspiré des shaders de Leo Zimmerman (\url{https://github.com/leozimmerman/ShadersLibrary}) et enfin les shaders pour les effets (\ref{s:effect}) est inspiré des exemples que fournis \textit{openFrameworks} (\url{https://github.com/openframeworks/openFrameworks/tree/master/examples/shader}).