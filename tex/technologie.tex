\section{Technologie}
\label{s:technologie}
L'environnement de développement du projet est composé du logiciel \textit{Visual Studio} et du framework \textit{openFrameworks}.
\subsection{Visual Studio}
Le développement de l'application se fait sur le système d'exploitation Windows pour toute l'équipe.
C'est donc naturellement que l'environnement de développement intégré \textit{Visual Studio} s'est imposé.
L'avantage est de pouvoir générer et configurer facilement un fichier projet pour que toute l'équipe de développement bénéficie de la même configuration de projet.
Il est aussi compatible avec le framework \textit{openFrameworks} qui nécessite l'installation d'une extension via le gestionnaire d'extension intégré et en ligne.
Aussi, \textit{openFrameworks} offre un générateur de projet compatible avec \textit{Visual Studio} permettant de créer une structure de projet vide et de gérer les add-ons dans les fichiers de configuration.

\subsection{openFrameworks}
L'application utilise le framework opensource \textit{openFrameworks} qui offre une interface haut-niveau avec la librairie de rendu 3D \textit{OpenGL}.
Ce framework offre plusieurs avantages dans le processus de développement dont le fait qu'il est distribué sur Windows, possède une communauté active qui fourni de la documentation et de nombreux tutoriels.
Ce aspect permet à l'équipe de développer dans un bon environnement avec des ressources disponibles. \\
\newline
Aussi, des add-ons sont utilisés pour aider à la création de l'interface graphique (\textit{ofxGui} et \textit{ofxInputField}) et l'importation de modèle 3D (\textit{ofxAssimpModelLoader}).

\subsection{Autres}
L'équipe a aussi utilisé le gestionnaire de version \textit{Git} pour le partage du code ainsi que \textit{Adobe Photoshop} pour le traitement des images. 